% Opcje klasy 'iithesis' opisane sa w komentarzach w pliku klasy. Za ich pomoca
% ustawia sie przede wszystkim jezyk i rodzaj (lic/inz/mgr) pracy, oraz czy na
% drugiej stronie pracy ma byc skladany wzor oswiadczenia o autorskim wykonaniu.
\documentclass[declaration,shortabstract, mgr]{iithesis}

\usepackage[utf8]{inputenc}

%%%%% DANE DO STRONY TYTUłOWEJ
% Niezaleznie od jezyka pracy wybranego w opcjach klasy, tytul i streszczenie
% pracy nalezy podac zarowno w jezyku polskim, jak i angielskim.
% Pamietaj o madrym (zgodnym z logicznym rozbiorem zdania oraz estetyka) recznym
% zlamaniu wierszy w temacie pracy, zwlaszcza tego w jezyku pracy. Uzyj do tego
% polecenia \fmlinebreak.
\polishtitle    {System kontroli obecności na podstawie elektronicznych legitymacji studenckich}
\englishtitle   {Students attendence verifing system with the use of student card}
\polishabstract {\ldots}
\englishabstract{\ldots}
% w pracach wielu autorow nazwiska mozna oddzielic poleceniem \and
\author         {Dawid Szczyrk}
% w przypadku kilku promotorow, lub koniecznosci podania ich afiliacji, linie
% w ponizszym poleceniu mozna zlamac poleceniem \fmlinebreak
\advisor        {dr Jakub Michaliszyn}
%\date          {}                     % Data zlozenia pracy
% Dane do oswiadczenia o autorskim wykonaniu
%\transcriptnum {}                     % Numer indeksu
%\advisorgen    {dr. Jana Kowalskiego} % Nazwisko promotora w dopelniaczu
%%%%%

%%%%% WLASNE DODATKOWE PAKIETY
%
%\usepackage{graphicx,listings,amsmath,amssymb,amsthm,amsfonts,tikz}
%
%%%%% WłaASNE DEFINICJE I POLECENIA
%
%\theoremstyle{definition} \newtheorem{definition}{Definition}[chapter]
%\theoremstyle{remark} \newtheorem{remark}[definition]{Observation}
%\theoremstyle{plain} \newtheorem{theorem}[definition]{Theorem}
%\theoremstyle{plain} \newtheorem{lemma}[definition]{Lemma}
%\renewcommand \qedsymbol {\ensuremath{\square}}
% ...
%%%%%

\begin{document}

\chapter{Wprowadzenie}

Celem niniejszej pracy jest zaprojektowanie użytecznego i niedrogiego w utrzymaniu systemu do kontroli obecności studentów na wykładzie z wykorzystaniem dostępnych legitymacji studenckich.\\
\indent Niemalże każde zajęcia na naszym instytucie zaczynają się od zebrania listy obecności. Odbywa się to w siermiężny sposób, ponieważ każdy ze studentów musi wpisać swoje dane na listę obecności.
Taka forma jest uciążliwy, zajmuje kilka pierwszych minut zajęć, a studenci się rozpraszają, ponieważ muszą sobie wzajemnie podawać listę obecności. Dziś, kiedy do ochrony danych osobowych przykłada się
coraz większą wagę, staromodne sposoby sprawdzania obecności są tym bardziej wątpliwe.\\
\indent Ze względu na to, że w trakcie moich studiów dostrzegałem powyższe braki oraz dlatego że bardzo zależało mi na tym, żeby w ramach pracy kończącej moje studia móc wykonać jakiś praktyczny projekt postanowiłem 
zaprojektować system, który możliwie usprawni proces weryfikacji obecności studentów na wykładzie. Zaproponowane przeze mnie rozwiązanie można rozszerzyć o wykorzystanie innych dokumentów identyfikujących
osobę, które są oparte na tej samej technologi, na której oparte są legitymacje studenckie.\\
\indent Zaproponowane przeze mnie rozwiązanie składa się z dwóch części:
\begin{enumerate}
\item Urządzenia opartego o platformę Arduino rozszerzoną o dodatkowe moduły niezbędne do odczytania potrzebnych informacji
\begin{itemize}
  \item Moduł RFID - odczytujący unikalny identyfikator z legitymacji studenckiej
  \item USB Host Shield - pozwalający na podłączenie zewnętrznej pamięci do Arduino i zapisanie zebranych informacji
  \item Wyświetlacz LCD + Buzzer - pozwalający na komunikację z użytkownikiem
  \item Zegar czasu rzeczywistego
\end{itemize}
\item Aplikacji internetowej pozwalającej na tworzenie i zarządzanie wykładami. Portal umożliwia utworzenia własnego użytkownika, do którego można przypisywać zajęcia, w ramach których można rejestrować obecności studentów
zebrane przy pomocy opisanego wyżej urządzenia.
\end{enumerate}
Funkcjonowanie całego systemy opiszę szczegółowe w rozwinięciu pracy.

\ldots

%%%%% BIBLIOGRAFIA

%\begin{thebibliography}{1}
%\bibitem{example} \ldots
%\end{thebibliography}

\end{document}
